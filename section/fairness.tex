\section{Bias and Fairness}

\subsection*{Goal}

Consider bias in the labeling, sample selection, task, features, loss functions and feedback loops.

\subsection*{Train-Test Mismatch}

Problem: distribution mismatch between old and new data.

\subsubsection*{Unsupervised Adaption (no label of new data)}
Importance sampling: $\E_{(x,y)\sim \mathcal{D}^{\text{new}}} l(y, f(x)) = \E_{(x,y)\sim \mathcal{D}^{\text{old}}} (\frac{\mathcal{D}^{\text{new}}(x,y)}{\mathcal{D}^{\text{old}}(x,y)}l(y,f(x)))$. Since $\frac{\mathcal{D}^{\text{new}}(x,y)}{\mathcal{D}^{\text{old}}(x,y)}$ is unknown, we could use approximations: $\Dist^{\text{new}} \propto \Dist^{\text{base}}p(s=0\mid x)$ and $\Dist^{\text{old}} \propto \Dist^{\text{base}}p(s=1\mid x)$. Therefore, $\frac{\mathcal{D}^{\text{new}}(x,y)}{\mathcal{D}^{\text{old}}(x,y)} = \frac{1}{p(s=1\mid x)}-1$. We can train a classifier to distinguish between old and new distribution and thus get the estimation of $p(s=1\mid x)$.

\subsubsection*{Supervised Adaption (some labels of new data)}

Use feature augmentation (separation) by forcing features into three components: shared, old-only (forced to be zero in the new dataset) and new-only (forced to be zero in the old dataset).